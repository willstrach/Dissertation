\documentclass[]{article}
\usepackage{cite}
\usepackage{hyperref}

\begin{document}

\begin{titlepage}
    \begin{center}
        \vspace*{1cm}
        
        \Huge
        \textbf{Using WordNet and Short-Term Memory for Contextual Disambiguation}
        \vspace{2cm}
        
        \Large
        \textbf{William T. F. Strachan}
        
        \vfill
                
        \vspace{0.8cm}
        
        \Large
        Word Count - PLACEHOLDER\\
        Supervisor - Dr Dimitar Kazakov\\
        Department of Computer Science\\
        University of York\\
        20 November 2016
        
    \end{center}
\end{titlepage}

\tableofcontents
\newpage
% ------------------------------ START OF DISSERTATION ------------------------------

\section{Literature Review}
\label{sec:LitReview}
In order to define a problem, we must first establish the previous works, so as to build upon them effectively. 
		
%--------------------- NLP

\subsection{Natural Language Processing}
\label{sec:NLP}
The study of natural language processing aims to allow a computer to understand natural language, and formulate a relevant response based upon its input. Within this, the problem of input text analysis has traditionally be broken down into smaller sub-problems\cite{NLPHandbook}:
\begin{itemize}
	\item Text Preprocessing
	\item Lexical Analysis
	\item Syntactic Parsing
	\item Semantic Analysis
\end{itemize}
The subsequent subsections will discuss each of these in more detail.

\subsubsection{Text Preprocessing}
\label{sec:TextPreprocessing}
Before any analysis can take place, the inputted raw text must be converted into a usable format. This, once again, can be broken down into multiple steps\cite{NLPHandbook}:
\begin{itemize}
	\item \textbf{Document Triage}

	\begin{itemize}
		\item Character encoding must be identified.
		\item The language can then be identified. %Language is most commonly found using one of two methods: Language can be identified by character set, in cases where language uses a unique alphabet, or by character frequency, in other cases.
		\item Non-useful data, such as images and html formatting must be removed.
	\end{itemize}
	
	\item \textbf{Text Segmentation}
	
	\begin{itemize}
		\item Individual words (tokens) must be separated from one another. % This is done using white space in space-delimited languages, and comprehensive lists in unsegemented languages
		\item Text Normalisation; replacing multiple equivalent tokens with one token (e.g. "Ave." and "Avenue").
		\item Identifying sentences, i.e. locating where a sentence begins and ends. 
	\end{itemize}
	
\end{itemize} 
% ########################
% #  UNFINISHED SECTION  #
% ########################


\subsubsection{Lexical Analysis}
\label{sec:LexicalAnalysis}
One word can have multiple forms, for example "judge" (the lemma) has the forms \{"judge", "judges", "judging", "judged"\} (morphological variants).  The job of Lexical Analysis is to replace all morphological variants of a word, with their corresponding lemma, a process known as stemming \cite{NLPHandbook}.

% In terms of ease of token identification, it can be seen conceptually how this process is beneficial. With that said, when we consider the amount of information in a word, it can similarly be realised that we lose information about the form of a word, for example plural/singular and tense.
\subsubsection{Syntactic Parsing}
\label{sec:SyntacticParsing}
Whwn derviving meaning from sentances, the grammatical structure can provide important insight. The Syntactic parsing technique extracts this infomation using two processes \cite{NLPAlmostFromScratch}: 
\begin{itemize}
	\item \textbf{Part-of-speech Tagging}
	
	\begin{itemize}
		\item Each word is given tags denoting their syntactic role (e.g. noun/adjective/verb).
	\end{itemize}		
	
	\item \textbf{Chunking}
	
	\begin{itemize}
		\item Noun phrases and verb phrases are detected and tagged using "begin chunk" and "inside chunk" labels
	\end{itemize}	
	
\end{itemize}

\subsubsection{Semantic Analysis}
\label{sec:SemanticAnalysis}
The semantics of a sentence, is the meaning given by it's tokens \cite{SemanticAnalysisAPracticalIntro}. The topic of semantic analysis will be expanded upon on in the proceeding subsections.

\subsubsection{Latent Semantic Analysis}
\label{sec:LSA}



%------------ Memory Models
\subsection{Psycholinguistics}
\label{sec:Psycholinguistics}
Language understanding is a problem which, it can be stated, is solved by the human brain. From this satatement, it can be derived that a computational solution could be effectively built around knowledge of the processes at work in the brain. The process of language comprehension can be described using the working memory model \cite{MemoryBaddeleyEysenkAnderson}.

According to Baddely et al. \cite{MemoryBaddeleyEysenkAnderson} there exist multiple, special purpose, memory structures within two main categories, the Short-term Store and the Long-term Store.  

\subsubsection{Long-term Store}
\label{LongTerm}
The long-term store (LTS) contains semi-permanent information. Within the LTS, there exist Explicit and Implicit memory structures. The contents of the Implicit memory describe skills and methods of doing things, whereas the Explicit memory contains factual information\cite{MemoryBaddeleyEysenkAnderson}. When considering these structures, it can be seen that Explicit memory is of greater interest in the context of NLP.

Within the Explicit memory exists knowledge of semantics \cite{MemoryBaddeleyEysenkAnderson}. The information held here not only defines concepts (meanings of word forms), but also their attributes and rules of use. In 1966, M. Quillain proposed a model of Semantic Memory \cite{SemanticMemoryQuillain}. The model consists of a graph of nodes, each representing a concept, connected by edges of differing types, each representing a different syntactic feature (for example, hypernym). 


\subsubsection{Short-term Store}
\label{ShortTerm}
The short-term store (STS) is a structure of limited capacity, used to store items for periods usually of no more than a few seconds \cite{MemoryBaddeleyEysenkAnderson}. In 1955, G. Miller, based upon previous experimental results, concluded that the size of the STS existed in the realm of 7$\pm$2 items of information \cite{SevenPlusMinusTwo}. 

In 1971, R. Atkinson and R. Shiffrin proposed a model of the STS \cite{ControlProcessesSTMAtkinson}. In this model, the STS can both send information to, and draw information from the LTS. Inputs from the sensory registers (memory structures holding information relating to inputs from senses) are also sent to the STS. Atkinson and Shiffrin proposed that, over time, the activation of items in the STS decreased; they went on to theorise that items could be lost from the STS, only when a new, more highly activated item could take its place. To counter this loss of activation, the authors discussed the control process, rehearsal. This process makes use of repetition to increase the activation of items in memory, decreasing their chance of loss. 

%------------ PREVIOUS WORK
\subsection{Previous Work}
\label{sec:PrevWork}


\newpage
\bibliography{references}
\bibliographystyle{IEEEtran}

\end{document}