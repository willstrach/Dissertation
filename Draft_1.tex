\documentclass[twocolumn]{article}
\usepackage{cite}
\usepackage{hyperref}

\begin{document}

\begin{titlepage}
    \begin{center}
        \vspace*{1cm}
        
        \Huge
        \textbf{Using WordNet and Short-Term Memory for Contextual Disambiguation}
        \vspace{2cm}
        
        \Large
        \textbf{William T. F. Strachan}
        
        \vfill
                
        \vspace{0.8cm}
        
        \Large
        Word Count - PLACEHOLDER\\
        Supervisor - Dr Dimitar Kazakov\\
        Department of Computer Science\\
        University of York\\
        20 November 2016
        
    \end{center}
\end{titlepage}

% ------------------------------ START OF DISSERTATION ------------------------------

\section{Literature Review}
\label{sec:LitReview}
In order to define a problem, we must first establish the previous works, so as to build upon them effectively. 

%--------------------- NLP

\subsection{Natural Language Processing}
\label{sec:NLP}
The study of natural language processing aims to allow a computer to understand natural language, and formulate a relevant response based upon its input. Within this, the problem of input text analysis has traditionally be broken down into smaller sub-problems\cite{NLPHandbook}:
\begin{itemize}
	\item Text Preprocessing
	\item Lexical Analysis
	\item Syntactic Parsing
	\item Semantic Analysis
\end{itemize}
The subsequent subsections will discuss each of these in more detail.

\subsubsection{Text Preprocessing}
\label{sec:TextPreprocessing}
Before any analysis can take place, the inputted raw text must be converted into a usable format. This, once again, can be broken down into multiple steps\cite{NLPHandbook}:
\begin{itemize}
	\item \textbf{Document Triage}

	\begin{itemize}
		\item Character encoding must be identified.
		\item The language can then be identified. %Language is most commonly found using one of two methods: Language can be identified by character set, in cases where language uses a unique alphabet, or by character frequency, in other cases.
		\item Non-useful data, such as images and html formatting must be removed.
	\end{itemize}
	
	\item \textbf{Text Segmentation}
	
	\begin{itemize}
		\item Individual words (tokens) must be separated from one another. % This is done using white space in space-delimited languages, and comprehensive lists in unsegemented languages
		\item Text Normalisation; replacing multiple equivalent tokens with one token (e.g. "Ave." and "Avenue").
		\item Identifying sentences, i.e. locating where a sentence begins and ends. 
	\end{itemize}
	
\end{itemize} 
% ########################
% #  UNFINISHED SECTION  #
% ########################


\subsubsection{Lexical Analysis}
\label{sec:LexicalAnalysis}
One word can have multiple forms, for example "judge" (the lemma) has the forms \{"judge", "judges", "judging", "judged"\} (morphological variants).  The job of Lexical Analysis is to replace all morphological variants of a word, with their corresponding lemma, a process known as stemming \cite{NLPHandbook}.

% In terms of ease of token identification, it can be seen conceptually how this process is beneficial. With that said, when we consider the amount of information in a word, it can similarly be realised that we lose information about the form of a word, for example plural/singular and tense.
\subsubsection{Syntactic Parsing}
\label{sec:SyntacticParsing}
Whwn derviving meaning from sentances, the grammatical structure can provide important insight. The Syntactic parsing technique extracts this infomation using two processes \cite{NLPAlmostFromScratch}: 
\begin{itemize}
	\item \textbf{Part-of-speech Tagging}
	
	\begin{itemize}
		\item Each word is given tags denoting their syntactic role (e.g. noun/adjective/verb).
	\end{itemize}		
	
	\item \textbf{Chunking}
	
	\begin{itemize}
		\item Noun phrases and verb phrases are detected and tagged using "begin chunk" and "inside chunk" labels
	\end{itemize}	
	
\end{itemize}

\subsubsection{Semantic Analysis}
\label{sec:SemanticAnalysis}
The semantics of a sentence, is the meaning given by it's tokens \cite{SemanticAnalysisAPracticalIntro}. The topic of semantic analysis will be expanded upon on in the proceeding subsections.

\subsubsection{Latent Semantic Analysis}
\label{sec:LSA}



%------------ Memory Models
\subsection{Memory Models}
\label{sec:MemModels}

%------------ PREVIOUS WORK
\subsection{Previous Work}
\label{sec:PrevWork}


\newpage
\bibliography{references}
\bibliographystyle{IEEEtran}

\end{document}